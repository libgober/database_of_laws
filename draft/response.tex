\documentclass{article}
\usepackage[margins=1in]{geometry}

\begin{document}
\flushleft
\noindent Manuscript Number: SDATA-23-01716 \newline
\noindent Manuscript Title: A Comprehensive Dataset of U.S. Federal Laws, 1789-2022  \newline 

Dear Dr. Jones and Anonymous Reviewers,

I thank you for the opportunity to revise the manuscript in light of this important feedback. I will respond to each reviewer in turn.

\subsection*{Reviewer 1}

\begin{itemize}

\item More Information On How To Use This Data

I thank the reviewer for this suggestion. The usage section is now significantly enhanced. Whereas previously the section had only one paragraph it now has four paragraphs that describes in much greater detail how the dataset might be used. In particular, I emphasize that one can use the dataset to conduct comparisons between some focus set to laws and those that are outside the focus (i.e. number of laws enacted during divided government versus unified government over time), and also as skeleton for constructing a dataset of laws with multiple labels where one wishes to understand the relationship between the labels (i.e. ``laws enacted during unified government'' and ``relative importance of the law"). Given the diversity of possible projects that scholars might have, the goal was not to describe every use case, but to give a general feel for the merging tasks that most scholars would typically follow.  I have clarified as well the confusing line 185-6.

\item How To Find Full Text of the Law

I thank the reviewer for this suggestion. At some point, I had considered including these texts in the data release, however for reasons I now elaborate in the draft this was a significant undertaking in its own right and something that I have left for later. The problem is basically that the public full texts are very limited in coverage, so one needs to use proprietary sources for the most part. These sources such as ProQuest and HeinOnline have digitized and scanned a lot of laws, but acquiring copies of these electronic versions in a comprehensive way is difficult. There is some possibility of using one government source, but it needs to be investigated.

Given that a comprehensive database of law \textit{texts} would be another project, the question as I understand it is how to go about finding the text of any particular law. This is not very difficult given that the citation system I use mirrors closely the table of contents of the original source documents, the statutes at large. I have typically included page numbers where available, so volume and page of statute at large is easily searched on HeinOnline and ProQuest. Where not available, one can use the table of contents of these volumes. 

\item Figure Improvement

\end{itemize}
\subsection*{Reviewer 2}

\end{document}

