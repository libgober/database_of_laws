\documentclass{article}
\usepackage[margin=1in]{geometry}

\begin{document}
\flushleft
\noindent Manuscript Number: SDATA-23-01716 \newline
\noindent Manuscript Title: A Comprehensive Dataset of U.S. Federal Laws, 1789-2022  \newline 

Dear Dr. Jones and Anonymous Reviewers,

I thank you for the opportunity to revise the manuscript in light of this important feedback. I will respond to each reviewer in turn.

\subsection*{Reviewer 1}

\begin{itemize}

\item More Information On How To Use This Data

I thank the reviewer for this suggestion. The usage section is now significantly enhanced. Whereas previously the section had only one paragraph it now has four paragraphs that describes in much greater detail how the dataset might be used. In particular, I emphasize that one can use the dataset to conduct comparisons between some focus set to laws and those that are outside the focus (i.e. number of laws enacted during divided government versus unified government over time), and also as skeleton for constructing a dataset of laws with multiple labels where one wishes to understand the relationship between the labels (i.e. ``laws enacted during unified government'' and ``relative importance of the law"). Given the diversity of possible projects that scholars might have, the goal was not to describe every use case, but to give a general feel for the merging tasks that most scholars would typically follow.  I have clarified as well the confusing line 185-6.

\item How To Find Full Text of the Law

I thank the reviewer for this suggestion. At some point, I had considered including these texts in the data release, however for reasons I now elaborate in the draft this was a significant undertaking in its own right and something that I have left for later. The problem is basically that the public full texts are very limited in coverage, so one needs to use proprietary sources for the most part. These sources such as ProQuest and HeinOnline have digitized and scanned a lot of laws, but acquiring copies of these electronic versions in a comprehensive way is difficult. There is some possibility of using one government source, but it needs to be investigated.

Given that a comprehensive database of law \textit{texts} would be another project, the question as I understand it is how to go about finding the text of any particular law. This is not very difficult given that the citation system I use mirrors closely the table of contents of the original source documents, the statutes at large. I have typically included page numbers where available, so volume and page of statute at large is easily searched on HeinOnline and ProQuest. Where not available, one can use the table of contents of these volumes. 

\item Showing Number of Laws in Full Data

I thank the reviewer for the suggestion. After some experimentation, I decided that adding some shading instead of a line helped preserve the figures original legibility, but also accomplished the goal of showing that this full data covers a lot more years of legislative activity.

\end{itemize}


\subsection*{Reviewer 2}

I thank the reviewer for the compliments toward the project and also for the question about ProQuest. I'll say a bit more bluntly here than I will in the draft: ProQuest likely does have this data or something very close to this on the backend powering their queries. Using ProQuest Congressional, it is possible to look up laws by page number and volume or volume and chapter. Doing at times returns multiple results, users are expected to click through and pick whichever one they are interested in, and then one can find details about the particular law such as Public Law number, date of passage, etc. \newline

Given that, it's pretty clear that ProQuest is sitting on a database with a few tables that allow this querying to work. At the same time, these data are not in a readily accessible tabular format like this dataset or the kinds social scientists are accustomed to loading in R, STATA, etc. And you really need that tabular data to do a lot of different work (i.e. making figures like my Figure 1, or running a regression of legislative importance on divided government). There is no pathway that I can see to using ProQuest Congressional to get such a database, or if there is a path it would be an immense amount of RA work individually querying their systems (expensive) or scraping their proprietary data (dicey). Moreover, they have done no technical validation or described their process. Finally, their coverage stops in 2014, for whatever reason. \newline 

I make these same points a bit more subtly in the draft by talking about transparency, verification, and reliance on public sources, but I still hope that they convey the difference between what this data is and what ProQuest Congressional offers. ProQuest Congressional is good for more targeted searches and in depth research on particular laws, but it's not the right tool for broader analysis across many laws. 
\end{document}

