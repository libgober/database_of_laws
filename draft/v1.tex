\documentclass[fleqn,10pt]{wlscirep}
\usepackage[utf8]{inputenc}
\usepackage[T1]{fontenc}
\usepackage{lineno}
\usepackage{subcaption}
\linenumbers


\title{A Comprehensive Dataset of U.S. Federal Laws, 1789-2022}

\author[$\dag$]{Brian Libgober}
\affil[$\dag$]{Northwestern University, Department of Political Science and School of Law, Evanston, IL, 60202, USA. brian.libgober@northwestern.edu}


\begin{abstract}
U.S. federal laws figure importantly in many research projects in political science, law, sociology, economics, and other disciplines. Despite their prominence, there is no authoritative, current, and comprehensive dataset of U.S. federal laws. In part, this is because such laws have been enacted over hundreds of years, resulting in a complicated patchwork of documents published in numerous and inconsistent formats. As a simplification, many scholars have relied upon selective lists of major legislative enactments or complete lists of relatively recent enactments. Here, I report on an effort to transparently and reproducibly assemble a complete database of US laws and their revision histories by combining data from HeinOnline, the Governmental Printing Office, and the National Archives and Records Administration. The result is a database of 49,746 laws spanning 1789 to 2022.
\end{abstract}

\begin{document}

\flushbottom
\maketitle
%  Click the title above to edit the author information and abstract

\thispagestyle{empty}

\section*{Background \& Summary}

%(700 words maximum) An overview of the study design, the assay(s) performed, and the created data, including any background information needed to put this study in the context of previous work and the literature. The section should also briefly outline the broader goals that motivated the creation of this dataset and the potential reuse value. We also encourage authors to include a figure that provides a schematic overview of the study and assay(s) design. The Background \& Summary should not include subheadings. This section and the other main body sections of the manuscript should include citations to the literature as needed. 

What is a law? The Oxford English Dictionary defines it as "a rule or system of rules recognized by a country or community as regulating the actions of its members and enforced by the imposition of penalties." \cite{law2008} So defined, laws in the United States can originate from many sources, including courts, the President, executive agencies, and so forth, both at the state and the national-level. Yet when scholars speak more formally about the set of "U.S. laws," they often intend to refer to a smaller set of authoritative documents. In particular, they frequently mean those federal laws enacted by the two Congressional chambers, potentially over the President's veto, through the legislative process outlined in Article 1 Section 7 of the U.S. Constitution. When Congress acts legislatively in this fashion, it may do so by issuing a variety of documents that go by different names, have different formal features, and follow different internal processes. These sub-genre distinctions can and indeed have changed over time. At present writing, the set of allowable legislation include bills, joint resolutions, concurrent resolutions, and simple resolutions. Of these, only bills and joint resolutions have the general and binding character that one typically associates with laws. And even then, there are exceptions and sub-divisions within these categories. ``Private bills" granting relief in one-off cases\footnote{For example, last year Congress enacted a private law allowing Alaska-woman Rebecca Trimble to obtain permanent resident status, despite the fact the paperwork her adoptive parents had used to bring her to the United States from Mexico had been defective.} are typically not what scholars have in mind when one refers to ``laws.'' Nor are joint resolutions approving constitutional amendments which are not ratified by the states really laws (e.g. the Equal Rights Amendment). Despite this esoteric mess of terminology, the set of enacted bills and joint resolutions is a relatively precise definition of what scholars \textit{should} mean when they talk about the set of U.S. federal laws, that is to say federal legislative enactments with a binding public character. Surprisingly, there is not, at least to the authors knowledge a transparent, fully comprehensive, and almost current database of these documents.

 Given the importance of U.S. federal lawmaking, it is surprising that no data source meets these goals of coverage and transparency. Yet it is worth detailing the numerous and painstaking data collection efforts that have come increasingly close. Some of the earliest existing data responsive to these criteria comes from a key debate in American political science about the consequences of divided government. In particular, scholars have wondered whether the state of unified government, where the President and Congress are controlled by the same party, is better or worse for legislative productivity than divided government, where these organs are not exclusively in controlled by the same political party. This debate was initiated by Mayhew (1991) \nocite{Mayhew1991}, who engaged this question quantitatively through a dataset of ``major'' legislative enactments. With slight simplification, Mayhew identified those pieces of legislation that were substantial through analyzing newspaper articles that summarized the achievements of the past legislative session, as we well as an examination of specialized historical works that retrospectively evaluated the importance of key pieces of legislation in a large number of policy areas. Mayhew's list of major enactments has been updated annually since its original publication in 1991 and is, at present writing, current from 1947 through 2022. Of course, this coverage of major legislative enactments is obviously limited as a database of all federal laws, both because it does not cover laws that fail to draw the attention of editorial pages, nor does it cover laws enacted during the New Deal or earlier.  Three issues implicit in Mayhew's pathbreaking data collection and analysis have continued to bedevil social scientists working in this area: transparency and reproducibility of inclusion or exclusion criteria, limited temporal coverage, and comprehensiveness within covered time-periods.

 There have been several follow-up efforts to Mayhew's project that are notable for their attempt to extend data on legislative enactments in various ways. Howell et al (2000) collect data on 17,633 public laws enacted between 1945 and 1994 and categorize them into four bins of importance, finding critical differences in terms of legislative productivity across unified and divided government according to legislative significance. \cite{Howell2000} Clinton and Lapinski (2006) enter the same debate by introducing an IRT model for estimating the importance of enacted laws between 1877 and 1994,\cite{clinton2006measuring} and have subsequently used these same data and estimates to reconsider questions about the interpretation of roll-call votes as a measure of political ideology \cite{clinton2008laws,lapinski2008,bateman2017}. The Clinton-Lapinski model of legislative importance leverages an even more comprehensive assembly than the Howell et al effort, comprising some 37,766 law going all the way back to the post-Civil War era.\footnote{At present writing this database was avialable at one of the author's websites, \url{https://my.vanderbilt.edu/joshclinton/data/}} Yet to the authors knowledge, this comprehensive dataset has not been updated to include laws enacted in the last two decades. Nor does it extend backwards into the first century of the U.S. Republic. As we shall see, that leaves about 25\% of all laws and almost 100 years of law-making uncovered.

Several other major research projects partially address the issue of currency and comprehensiveness, but neither in a completely satisfactory way. It is also worth emphasizing that these projects all use somewhat different approaches to identifying laws, which are mostly but not perfectly inter-operable. The Policy Agendas Project, for example, is a long-standing collaborative research effort aimed at building multiple datasets for tracking and comparing which issues manage to attract the attention of government across. It covers many national and subnational systems, including the United States federal government. In particular, their database of bills covers "more than 400,000 bills introduced by the U.S. Congress." Importantly, they also have coded each bill according to the subject-matter coding system of the larger Policy Agenda Project. Their focus on covering the subject matter of lawmaking is an important undertaking, although arduous, and it is unsurprising given the difficulties of this undertaking that they have only managed to cover the period 1947-2022. 

Another project with relevant and similar prior data collection is due to Ansolabehere, Palmer, and Schneer.\cite{ansolabehere_palmer_schneer_2016,ansolabehere2018divided} Their data collection, which was conducted under the auspices of an undergraduate course at Harvard called ``What Has Congress Done," crowd-sourced the task of identifying significant legislation to students, each of whom was responsible for producing a list of significant legislation enacted under each of 22 potentially assigned decades. The researchers provided a list of secondary sources, as well as guidance and quality control efforts. Similar to Mayhew, their criteria for inclusion were (a) was the law considered significant at the time of enactment, and (b) does the law seem significant in historical perspective. Because of their differing methods, Mayhew and Ansolabhere, Palmer and Schneer do not completely overlap in the 1947-2022 period as databases of significant legislation. Although the temporal coverage for the Ansolabehere, Palmer, and Schneer dataset is very long, it contains a small percentage of all laws enacted by the U.S. Congress. None of the so-far mentioned sources, therefore, comes anywhere near a comprehensive dataset of U.S. laws. 


%There are several reasons why it is desirable to have a more complete dataset of U.S. lawmaking than has previously existed. Temporarlly In particular, minor legislation is usually major to somebody, otherwise why would someone go to the trouble of passing it? While focusing on the big pieces of legislation is perhaps appropriate for thinking about how party control effects the ability of collective government to solve the big problems, a more granular perspective is really necessary to think about how special interests may whittle away the gains made by broad coalitions through smaller pieces of legislation, or perhaps how policy fails to make the necessary incremental adjustments to keep with the times. There is notable inconsistency in what laws are included as significant using all these methods. The lack of a unified and complete dataset also frustrates progress of the discipline, as these authors do not necessarily use the same system to refer to particular laws, making U.S. law datasets used by social scientist far less inter-operable than they should be. The lack of inter-operability frustrates project

This project seeks to build a comprehensive and easily updated list of U.S. laws and their revision histories based on publicly available sources. The primary source we leverage is the \emph{Statutes at Large}, which has been published continuously since 1845, and whose early issues contain the earliest legislative enactments. The most recent issue of the \emph{Statutes at Large} is from 2017, and it is published with about a six year delay. Even though the \emph{Statutes at Large} are not physically or electronically available for the last six years, the National Archives and Records Administration provides citations to page numbers almost immediately.\footnote{For example, on September 25, 2023 it was possible to get the statute at large citation of a law enacted only three days prior \url{https://web.archive.org/web/20230925173228/https://www.archives.gov/federal-register/laws/current.html}} 

Although the \emph{Statutes at Large} are well-known to scholars, this source have not until recently been ``born digital." As a result, when electronic data on these laws are available, they have typical issues of digitized text data, despite the very high importance of these documents. Indeed, it is somewhat surprising that the U.S. government does not make electronic copies of the \emph{Statutes at Large} available prior to 1951,\footnote{https://www.govinfo.gov/app/collection/statute/2016} despite the fact that there were 64 earlier volumes containing laws that in many cases have not been amended. In legal practice, not having these documents digitized usually works out well enough, because of ongoing public and private efforts to develop codices of laws, however there are occasional issues of interpretation where returning to the actual text of the law in the \emph{Statutes at Large} is necessary. In addressing such esoteric issues, legal scholars and lawyers often leverage use an electronic database called HeinOnline, which has digitized all these paper volumes in their entirety. Fortunately, they also provide meta data on the tables of contents of these volumes. For years in which the government-provides meta-data about laws through its Governmental Printing Office, we can rely on this meta-data to construct a list of laws. For older years that the GPO has not yet reached, we rely on the meta-data provided by HeinOnline. For more recent years where the \emph{Statutes at Large} do not yet exist, we rely on data on the National Archives website to supplement and bring our dataset to currency.\footnote{https://www.archives.gov/federal-register/laws. Per that site, ``After the President signs a bill into law, it is delivered to the Office of the Federal Register (OFR)[, a division of the National Archives], where editors: assign a Public Law Number[,] prepare it for publication as a Slip Law[,] include it in the next edition of the United States \emph{Statutes at Large}.''}

 Our dataset covers laws enacted from 1789 to 2022, the year prior to publication, and contains 49,746 entries. A Github repository provides linked to this publication provides code to update the set of laws to the present.


\section*{Methods}

We rely on three primary sources for our comprehensive database of U.S. laws: the oldest meta data for the U.S. \emph{Statutes at Large} disseminated via HeinOnline, similar and more recent meta data through the Governmental Printing Office, and finally the last six years of law-making as described by the National Archives' website.

We began our data collection by scraping the table of contents of each volume of the \emph{Statutes at Large}, excepting volumes 6-8 which contain a collection of early treaties and private laws. Importantly, examination of these tables of contents quickly reveal weaknesses in many approaches to uniquely identifying laws. In particular, the public law numbers used today were not used consistently prior to the 20th century. There were also periods in the 20th century where public law numbers were repeated across sessions of the same convening of Congress, as opposed to more recent practice where the new session of a particular Congress starts counting where it left off in the last session. Citing pages and volumes of the \emph{Statutes at Large}, probably the most common practice in ordinary government and legal use, does not effectively disambiguate laws because it is not uncommon for two or even three laws to share the same starting page. Presumably, individuals who care about the text of a particular law know how to fill in the gaps caused by this ambiguity, however for the purposes of building \textit{databases} this is clearly not ideal. Disambiguating based on the names of the laws is possible, but potentially mistake prone as different sources may use different capitalizations, punctuation, and spacing, short versus long names, and other issues typical of text data. 

These caveats aside, examination of the published pages of the \emph{Statutes at Large} as well as the table of contents reveals that laws \textit{are} identified using an (almost perfectly) precise numerical citation systems, however the exact system that works depends on the era. In particular,

\begin{itemize}
\item The currently practiced identification system ("Late regime") begins on Jan 7, 1959 with the 86th Congress. In this system, laws are identified by the number of the Congress and the law number within Congress. An example is ``Public Law 105-89, An Act: to promote the adoption of children in foster care.'' There is only one Public Law 105-89, it uniquely refers to that titled law, and the \emph{Statutes at Large} has published the text of the law with that identifier.
\item Initially, laws were identified by a different system ("Early Regime"). From the 1st through the 56th Congress, which concluded March 3, 1901, laws were identified in the \emph{Statutes at Large} by chapter as well as Congress and session (e.g. ``Chapter 15, 56 Congress, Session 1, An Act: Relating to Cuban vessels."). All laws in the \emph{Statutes at Large} are published with this information during the early period. The Early Regime and Later Regime are not interoperable, as the published pages of the \emph{Statutes at Large} in the early period do not provide public law numbers and the published pages in later period do not provide chapters. The early and later periods use fundamentally different citation systems.
\item For the 57th through 85th congress, the \emph{Statutes at Large} uses both systems, but not always consistently, and not consistently in the same way as they would before or after. Indeed, a key difficulty in this period is that chapters and public laws are sometimes recycled within one Congress over multiple sessions of a particular Congress.\footnote{And worse, in just a few instances, the same session. Indeed, PL 65-246 refers to both a law ``Providing for the transportation from the District of Columbia of governmental employees whose services no longer are required." and also a law intended ``To authorize the sale of certain lands to school district numbered twenty-eight, of Missoula County, Montana."} The law identification system that superficially seems to be inter-operable with both the prior and subsequent systems does not really work very well with either.  Careful attention to chapters, public laws, and sessions does disambiguate nearly all laws, however these details are very particular and subtle. Caution is needed.
\end{itemize}

 Following these citation practices, we used a rule-based parser to extract key pieces of information from scraped meta-data and build a database of laws. Technical validation efforts revealed that there were occasional errors of transcription in our source data, for example mis-identification of chapters associated with laws or including page numbers that were not correct. Those errors that were found in the source data we corrected to conform with the text image captured on the HeinOnline site, although our guess is that there are others which have not been caught. Such issues highlight the needs for \textit{transparency} and \textit{reproducibility} in the process of data collection and data correction, as it is unlikely that any effort that lack either will avoid such issues. 
 
% The Methods should include detailed text describing any steps or procedures used in producing the data, including full descriptions of the experimental design, data acquisition assays, and any computational processing (e.g. normalization, image feature extraction). See the detailed section in our submission guidelines for advice on writing a transparent and reproducible methods section. Related methods should be grouped under corresponding subheadings where possible, and methods should be described in enough detail to allow other researchers to interpret and repeat, if required, the full study. Specific data outputs should be explicitly referenced via data citation (see Data Records and Citing Data, below).

% Authors should cite previous descriptions of the methods under use, but ideally the method descriptions should be complete enough for others to understand and reproduce the methods and processing steps without referring to associated publications. There is no limit to the length of the Methods section. Subheadings should not be numbered.

% \subsection*{Subsection}

% Example text under a subsection. Bulleted lists may be used where appropriate, e.g.

% \begin{itemize}
% \item First item
% \item Second item
% \end{itemize}

% \subsubsection*{Third-level section}
 
% Topical subheadings are allowed.

\section*{Data Records}

The primary dataset we provide is publicly available at https://osf.io/qa289. Additionally, a Github repository that was used to construct the dataset will be made available upon publication. 

% The Data Records section should be used to explain each data record associated with this work, including the repository where this information is stored, and to provide an overview of the data files and their formats. Each external data record should be cited numerically in the text of this section, for example \cite{Hao:gidmaps:2014}, and included in the main reference list as described below. A data citation should also be placed in the subsection of the Methods containing the data-collection or analytical procedure(s) used to derive the corresponding record. Providing a direct link to the dataset may also be helpful to readers (\hyperlink{https://doi.org/10.6084/m9.figshare.853801}{https://doi.org/10.6084/m9.figshare.853801}).

% Tables should be used to support the data records, and should clearly indicate the samples and subjects (study inputs), their provenance, and the experimental manipulations performed on each (please see 'Tables' below). They should also specify the data output resulting from each data-collection or analytical step, should these form part of the archived record.

\section*{Technical Validation}

% This section presents any experiments or analyses that are needed to support the technical quality of the dataset. This section may be supported by figures and tables, as needed. This is a required section; authors must present information justifying the reliability of their data.

We offer two primary approaches to support the technical validation of our dataset. The first is to compare the total counts of laws by Congress against total counts previously published by social scientists. The second validation exercise involves showing that we cover well a particularly comprehensive dataset of laws published by the Office of Revision Counsel, the agency within Congress  responsible for preparing the U.S. Code.

The reference counts for total laws that we rely on comes from Appendix F reported in Galloway and Wise's History of the House of Representatives\nocite{gallowaywise}. It was the same set of reference counts previously used for validation and analysis by Ansolabhere, Palmer, and Schneer.\cite{ansolabehere_palmer_schneer_2016} Our definition of law is slightly more capacious than Galloway and Wise's, in the sense that we also consider "Joint Resolutions" laws, and the authors admit their own counts combine acts and resolutions (the latter of which are not, in our view, best understood as laws) from the 77th Congress and after. Nevertheless, it is possible to subset our data in such a way as to include the same set of documents as Galloway and Wise cover, and thereby constitute an appropriate test of technical validity. Figure \ref{fig:totals} compares the counts for this key subset of our data. Note that in total Galloway and Wise count 23,155 laws, while we would peg the figure at 23,152, or three laws fewer ($\approx 0.13\%$ of the total). The true discrepancy between our counts and Appendix F is somewhat greater, because in some years our counts are higher than Galloway \& Wise's and in some years lower. However, in all but six Congresses, our counts are identical. Summing up the absolute value of the difference in counts in these six Congresses, we arrive at a total difference in counts of 11, again a very small difference on a percentage basis. An undergraduate research assistant further investigated the difference in counts of laws in  the 11th, 15th, and 34th Congress. In each case, the number of laws that the research assistant counted in that Congress was the same as the number we provided, and we were unable to determine why there was a discrepancy between the Galloway and Wise total and our own. 

Our second validation exercise works by attempting to match our database of laws against a particularly large collection of U.S. laws. The dataset we use is the table of official popular names for Congressional enactments as produced by the Office of Revision Counsel.\footnote{https://uscode.house.gov/popularnames/popularnames.htm} This table catalogues in its rows a set of 13,170 popular or short-names for official legal acts. The table's rows have entries such as ``Obamacare," which directs readers to see another row on the ``Patient Protection and Affordable Care Act." In the row corresponding to the ``Patient Protection and Affordable Care Act," the table provide a citation ``Pub. L. 111-148, Mar. 23, 2010,124 Stat. 119." Although the table contains many hundreds of rows cross-referencing entries, which are not directly useful for testing, the source nevertheless can be used to construct a very strong validation test of our source data. In particular, it has very long temporal coverage, especially as compared with most datasets of important legislation, and it has very deep coverage, as it cites many thousands more laws than most other datasets. The main alternative contender with equivalent temporal coverage, the crowd-sourced dataset produced by Ansolabhere, Palmer, and Schneer,\cite{ansolabehere_palmer_schneer_2016} has many thousands less laws and follows considerably less consistent citation practices than the Office of Revision Counsel.

That said, while the Popular Names Table is a very good source for validating, there are a few difficulties. First, and perhaps most importantly, the popular names table includes names for specific \textit{pieces} of legislation that are part of larger legislative enactments. For example, the Biggert-Waters Flood Insurance Reform Act of 2012 refers just to subtitle A of Title II of Division F of a 584-page highway bill enacted in the same year (Public Law  112-141). While our database of laws does include 112-141 as a law, it does not include titles and sub-titles like the Biggert-Waters Act, so such rows need to be excluded from our matching exercise. Secondly, even after screening out these subpart the popular names table also includes a number of suspicious entries. These involve issues such as two names referring to the same public law number but claiming different enactment dates. This issue impacts some 745 entries, which is a considerable number, and there are also other similarly suspicious issues involving citations. Because of these and several other issues, and also because the number of laws with popular names is considerably smaller than the number of laws, we view the key test of our dataset as what proportion of the \textit{unproblematic entries} in the popular names table are matchable against our complete enumeration of U.S. laws. We find that out of 7,315 cleanly cited laws in the popular names table, we are able to match 7,229 to our list of U.S. laws (98.8\%). Part of the reason we are not able to do even better is that the popular names table does not always disambiguate page citations to the statute at large by referring to chapter or other suitable citation. As a result, if there are multiple laws on a particular page we cannot automatically tell which popular name goes with which law on that page. If we count those popular names matching to more than one law as a ``hit" in our query, it turns out we find 7,284 matches, which is to say 99.6\% of all the laws we search for can be found in our database. 

\section*{Usage Notes}  \label{sec:usage}

While we provide a numeric identifier that can be used for referring to particular rows of our dataset, we recommend that future researchers use identification for laws that reflect greater attention to the inconsistencies in citation practice over time. In particular,  we recommend matching efforts that use piece-wise rules as outlined in the methods section above and implemented via special merging functions in the git repository. Beginning with the 86th Congress laws should be matched using the public law number. Prior to the 56th Congress, laws should be matched by chapter number, Congress, and session. Finally, particular care needs to be paid in the matching of laws enacted between the 57th and 85th Congress. Generally, if public laws and chapters will be used to refer to laws in a dataset that is not exclusively relating to recent laws, one \textit{must} also include the Congressional session (or equivalently include the date of enactment, from which one can infer these details). 



% The Usage Notes should contain brief instructions to assist other researchers with reuse of the data. This may include discussion of software packages that are suitable for analysing the assay data files, suggested downstream processing steps (e.g. normalization, etc.), or tips for integrating or comparing the data records with other datasets. Authors are encouraged to provide code, programs or data-processing workflows if they may help others understand or use the data. Please see our code availability policy for advice on supplying custom code alongside Data Descriptor manuscripts.

% For studies involving privacy or safety controls on public access to the data, this section should describe in detail these controls, including how authors can apply to access the data, what criteria will be used to determine who may access the data, and any limitations on data use. 

\section*{Code availability}

All code used to create the dataset from original sources, is available through a Github repository made public upon publication. Also included in the repository are code and data used to validate the dataset as well as generate the figure. \footnote{\url{https://github.com/libgober/database_of_laws}}


% For all studies using custom code in the generation or processing of datasets, a statement must be included under the heading "Code availability", indicating whether and how the code can be accessed, including any restrictions to access. This section should also include information on the versions of any software used, if relevant, and any specific variables or parameters used to generate, test, or process the current dataset. 

\bibliography{biblio}

% \noindent LaTeX formats citations and references automatically using the bibliography records in your .bib file, which you can edit via the project menu. Use the cite command for an inline citation, e.g. \cite{Kaufman2020, Figueredo:2009dg, Babichev2002, behringer2014manipulating}. For data citations of datasets uploaded to e.g. \emph{figshare}, please use the \verb|howpublished| option in the bib entry to specify the platform and the link, as in the \verb|Hao:gidmaps:2014| example in the sample bibliography file. For journal articles, DOIs should be included for works in press that do not yet have volume or page numbers. For other journal articles, DOIs should be included uniformly for all articles or not at all. We recommend that you encode all DOIs in your bibtex database as full URLs, e.g. https://doi.org/10.1007/s12110-009-9068-2.

\section*{Acknowledgements} (not compulsory)

% Acknowledgements should be brief, and should not include thanks to anonymous referees and editors, or effusive comments. Grant or contribution numbers may be acknowledged.

The author would gratefully acknowledge the help of a number of research assistants, especially Jamie Gall, Kakshak Porwal, and Christina Leung, as well as funding through the Institute of Policy Research and Department of Political Science at Northwestern and the School of Global Strategy and Public Policy at University of California San Diego. 

\section*{Author contributions statement}

B.L. was the sole contributor.

% Must include all authors, identified by initials, for example:
% A.A. conceived the experiment(s), A.A. and B.A. conducted the experiment(s), C.A. and D.A. analysed the results. All authors reviewed the manuscript. 

\section*{Competing interests} (mandatory statement)

The author declares that he has no competing interests related to the research, authorship, or publication of this manuscript, and no conflicts of interest, financial or otherwise, exist.
% The corresponding author is responsible for providing a \href{https://www.nature.com/sdata/policies/editorial-and-publishing-policies#competing}{competing interests statement} on behalf of all authors of the paper. This statement must be included in the submitted article file.

\section*{Figures \& Tables}


\begin{figure}[h]
  \centering
  \caption{Number of Acts of Congress by Meeting of Congress.}
  \includegraphics[width=\linewidth]{draft/figures/counts_of_laws.png}
    \label{fig:totals}
\end{figure}


% Figures, tables, and their legends, should be included at the end of the document. Figures and tables can be referenced in \LaTeX{} using the ref command, e.g. Figure \ref{fig:stream} and Table \ref{tab:example}. 

% Authors are encouraged to provide one or more tables that provide basic information on the main ‘inputs’ to the study (e.g. samples, participants, or information sources) and the main data outputs of the study. Tables in the manuscript should generally not be used to present primary data (i.e. measurements). Tables containing primary data should be submitted to an appropriate data repository.

% Tables may be provided within the \LaTeX{} document or as separate files (tab-delimited text or Excel files). Legends, where needed, should be included here. Generally, a Data Descriptor should have fewer than ten Tables, but more may be allowed when needed. Tables may be of any size, but only Tables which fit onto a single printed page will be included in the PDF version of the article (up to a maximum of three). 

% Due to typesetting constraints, tables that do not fit onto a single A4 page cannot be included in the PDF version of the article and will be made available in the online version only. Any such tables must be labelled in the text as ‘Online-only’ tables and numbered separately from the main table list e.g. ‘Table 1, Table 2, Online-only Table 1’ etc.

% \begin{figure}[ht]
% \centering
% \includegraphics[width=\linewidth]{./assets/stream}
% \caption{Legend (350 words max). Example legend text.}
% \label{fig:stream}
% \end{figure}

% \begin{table}[ht]
% \centering
% \begin{tabular}{|l|l|l|}
% \hline
% Condition & n & p \\
% \hline
% A & 5 & 0.1 \\
% \hline
% B & 10 & 0.01 \\
% \hline
% \end{tabular}
% \caption{\label{tab:example}Legend (350 words max). Example legend text.}
% \end{table}

\end{document}